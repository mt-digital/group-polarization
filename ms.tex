\documentclass[a4paper,12pt]{article}
\usepackage[utf8]{inputenc}
\usepackage[english]{babel}
\usepackage{authblk}
\usepackage{booktabs}
\usepackage{apacite}
\usepackage{graphicx}
\usepackage{mathptmx}
\usepackage[singlespacing]{setspace}
\usepackage[headheight=1in,margin=1in]{geometry}
\usepackage{fancyhdr}
\usepackage{amsmath}
\DeclareMathOperator*{\argmin}{arg\,min} % thin space, limits underneath in displays
\renewcommand{\headrulewidth}{0pt}
\pagestyle{fancy}
\chead{%
  $6$$^{th}$ International Conference on Computational Social Science IC$^{2}$S$^{2}$\\
  July 17-20, 2020, MIT, Cambridge, Massachusetts
}

% \graphicspath{{Figures/}}

\title{Empirical and computational studies of group polarization}

\author[]{} % Please leave Author-field blank for blind review and remove information that may identify the author(s)
 
\date{}

\begin{document}

\maketitle
\thispagestyle{fancy}

\vspace{-6em}
\begin{center}
\textbf{\textit{Keywords: Opinion dynamics, model validation, psychology, agent-based modeling, ???}}
\newline
\end{center}

\section*{Extended Abstract}

There are many opinion dynamics models that make different assumptions about
the social influence process, but it remains unclear which assumptions are
empirically valid \cite{Flache2017}. To make progress, one approach we take
here is to examine a particular social psychological phenomenon, ``group
polarization''. This is an essential component of ``bi-polarization'' of the
sort observed in U.S. politics, for example, but not the same. Instead, group
polarization refers to the phenomenon where a group's average opinion on 
some topic often leads to greater average opinion extremity 
after the group discusses the topic, compared to pre-discussion opinions.
Group polarization has been observed with such regularity, 
\citeA{Sunstein2002} suggested this
observed regularity be regarded as a scientific law. One approach to 
modeling group polarization is based on a model of ``persuasive arguments'', 
which seem to qualitatively explain experiments in one example \cite{Mas2013}.
However, no work has been done to calculate key parameters used in that model,
and the persuasive argument model component increases opinion dynamics model
complexity, which are already quite complex.

In this paper I present an alternative approach that enables direct 
model comparison to data, including the calculation of meaningful parameters.
To do this I modify an existing model of opinion dynamics developed by
\citeA{Flache2011} and further investigated by \citeA{Turner2018}. The only
modification is the introduction of two model parameters which will represent
the strength of negative influence and the stubbornness of extreme opinions.
The results are the calculation of these parameters for a number of 
experimental settings using published research data to initialize our models
and fit the free parameters that determine optimal model outcomes found by
minimizing an error term over parameter space. Our results will partly 
support the claim by \citeA{Mas2013} that negative influence is not necessary
for group polarization. However, our results also show that persuasive arguments
and homophily are not necessary for group polarization. 
Furthermore, our method can be applied broadly to validate model assumptions 
and calculate model parameters for empirical studies.

(DATA SOURCES 1 par) For this preliminary work, we focus on two studies with
fairly clear results. The first is a study of changes in racial attitudes among
discussants in three groups who scored high, medium, or low on a racial prejudice
pre-test \cite{Myers1970}. The second study examined shifts to more conservative
or liberal views about feminism among discussants who were separated at the outset 
into ``chauvinist'' or ``feminist'' groups based on their initial responses to
a questionnaire \cite{Myers1975a}. In the first experiment, there were four
discussants per group, and in the second experiment, there were between 22 
in the chauvinist group and 26 in the feminist group. The opinion scale in 
the first experiment was a nine-point scale, from -8 to +8 with -8 indicating 
```white racism' is not responsible'' for bad conditions for African-Americans
in US cities, and +8 indicates certainty that ``white racism'' is responsible.
The opinion scale in the second study is a seven-point scale ranging from -3,
indicating maximally conservative views on feminism, and +3 representing
maximally liberal views on feminism, as determined through a six-item
questionnaire. To obtain our computational results, these scores are scaled to
lie between -1 and 1. When we calculate means, we first bin respondent scores
into one of nine or seven bins, assign the appropriate categorical score, 
and then calculate the mean. In the nine-point scale, for example, if the
opinion fell between $\pm \frac{1}{18}$, this opinion would be given a zero;
if the opinion were between -1 and $-1 + \frac{1}{9}$, the opinion would be
-8, and so on. The data for these studies and selected experimental conditions
are given in Table~\ref{tab:results} along with the results of model-fitting
to parameterize our model for each individual condition.

% \begin{itemize}
%   \item \citeA{Mas2013}: Still need to answer the question of whether they 
%     calculate model parameters or just do a bait and switch and fit a 
%     regression model that ``looks like'' the ABM. 
%   \item \citeA{Sunstein2002}: References to a number of studies. See p. 178
%     \begin{itemize}
%       \item Zuber et al "Choice shift and group polarization" J Pers. and 
%         Soc. Psych. 1992. Group polarization in Germany.
%       \item Abrams "Knowing what to think by knowing who you are" 
%         Brit J. of Soc Psych 1990. Group polarization in New Zealand.
%       \item Myers "Discussion-induced attitude polarization" Hum. Relations 1971.
%         Moderately pro-feminist groups become more stronglly so after discussion.
%       \item Myers and Bishop "Enhancement of dominant attitudes in group
%         discussion" J. Pers. Soc. Psych 1971. More racist whites increase
%         negative responses to whether racism is responsible for bad conditions
%         in American cities; less racist whites become more positive in their
%         answer after discussion.
%       \item R. Brown \emph{Social Psychology} 2nd. Ed. 1985 -- ILL requested:
%         Citizens of France become more critical of the US and its intentions
%         with respect to economic aid.
%     \end{itemize}
%   \item \citeA{Krizan2007}: Showed ``self-categorization'', or equivalently
%     ``other-categorization'', is not necessary for group polarization in
%     their behavioral studies. I believe self-categorization is
%     equivalent to negative influence, but need to argue this.
% \end{itemize}

Method (2 par):
I sketch the opinion update process that incorporates categorization effects
and the stubbornness of extreme opinions, and results in group polarization,
leaving out some details. In the model of \citeA{Flache2011},
agent $i$'s opinion on cultural issue $k$ at time $t$, $s_{ik,t}$, 
is given by $s_{ij,t} = s_{ij,t-1} + \Delta s_{ij,t} \left ( 1 - |s_{ij,t-1}| \right )$.
$\Delta s_{ij,t}$ is determined through weighted social influence. Opinions on
each cultural issue are bounded arbitrarily between -1 and 1. Leaving aside
many details for now, we propose that we can modulate the relative attractiveness
of extreme opinions by adding a parameter, $\alpha$, to the update equation
to make it 
\[
  s_{ij,t} = s_{ij,t-1} + \Delta s_{ij,t} \left ( 1 - |s_{ij,t-1}|^{\alpha} \right ).
\]
There are a number of theoretical reasons to assume that, after interaction,
agents would update their opinions to be more extreme, including the most
extreme opinions typically being the most convincing (REF) and individuals
being more comfortable taking on and expressing more extreme opinions if 
they know they have the social support of others with extreme opinions. 
We obtain our results by calculating $\alpha$ for five different experimental
conditions from the two studies introduced above. With $\delta_{obs}$ being
the observed mean opinion shift in each experimental condition and 
$\delta_{calc}(\alpha)$ being the calculated opinion shift, which is assumed to
be a function of $\alpha$, we calcualte the best-fit $\alpha$ as the $\alpha$
that minimizes the squared error, i.e.\
\[
  \argmin_{\alpha} \left( \delta_{obs} - \delta_{calc}(\alpha) \right )
\]

% Positive assimilation,
% where agents become more alike in opinion after interaction, occurs when the
% Manhattan distance between opinions is less than 1. Negative assimilation 
% occurs when the distance is greater than 1. With $d_{ij,t}$ as the Manhattan distance
% between opinion vector $i$ and $j$, the force of this attraction (positive
% assimilation) or repulsion (negative assimilation) scales as $w_{ij,t} = 1 - d_{ij,t}$.
% This incorporates ``biased assimilation'' where more similar opinions are
% more attractive and more dissimlar opinions are more repulsive. 


% We propose a modification with the addition of a parameter that measures the
% effect of in-group versus out-group bias, where group membership is determined
% by opinion distance \cite{Blau1977}. For this operationalization of 
% categorization effect, the parameter will be $\gamma$ and introduced as a power 
% in the weighting function, $w_{ij,t} = (1 - d_{ij,t})^\gamma$---(BUT SHOULD IT
% BE $1/\gamma$ and switch the following explanation???)
% Since $|w_{ij,t}| \leq 1$, lesser $\gamma$ will amplify
% the influence of less extreme opponents and less agreeable allies, 
% and greater $\gamma$ will attenuate the influence of less extreme opponents and less 
% agreeable allies. 

% By running models with either $\alpha$ and/or $\gamma$ as parameters to be fit,
% we can run specific instances of the models with some $(\alpha, \gamma)$ pair,
% compare the outcome measure of shift from initial mean to final mean opinion,
% and update the parameters to find the best fit between the shift observed
% empirically and the shift observed in the averaged model runs. Using a 
% gradient descent method we can find the $(\alpha, \gamma)$ combination that
% produces the best fit to empirical results. For these experiments we set the
% network to be (-TBD-).


(RESULTS 1-3 par): 

\begin{table}
  \centering
  \caption{Observed means and shift in means from empirical studies and the
  fitted parameter that minimized the error between observed and predicted
  shift, $\alpha$. Myers \& Bishop (1970) used a nine-point scale and 
  Myers (1975) used a seven-point scale. The best-fit $\alpha$ exponent is given
  in the rightmost column.}
  \label{tab:results}
  \vspace{0.5em}
  \begin{tabular}{crcccc}
    Study     & Condition     & $\mu_{t=0}$ & $\mu_{t=T}$ & $\delta_{obs}$ & $\alpha_{calc}$ \\
    \toprule
    Myers \& Bishop 1970 & Low prejudice & 2.94 & 3.41 & +.47 & 1.25 \\
                         & Medium prejudice & 1.3 & 0.67 & -0.64 & 0.4 \\
                         & High prejudice & -1.7 & -3.01 & -1.3 & 1.5 \\
                         \midrule
    Myers 1975 & Chauvinists & -1.11 & -1.14 & -0.02 & 0.1 \\
               & Feminists & 0.78 & 1.72 & 0.95 & 1.3 \\
               \bottomrule
  \end{tabular}
\end{table}



% (DISCUSSION 2-3 par)
% \begin{itemize}
%   \item Zoom out. How does this impact the larger project not just of validating
%     the assumptions of opinion dynamics models but also using them as models
%     for data analysis?
%   \item How can this approach be applied to other empirical data sources and
%     other components/levels of opinion dynamics assumptions/outcomes?
%     E.g.\ \emph{we believe this approach could be valuable to apply to multiple levels 
%     of opinion dynamics, from cognitive assumptions to the outcome variable of
%     the distribution of opinions of the population.} Follow up with two examples:
%     \begin{itemize}
%       \item Neuro-level validation of assimilation and homophily assumptions (ir-neuro).
%       \item Population-level outcomes as illustrated in \citeA{Mason2018}.
%     \end{itemize}
%   \item Connection to ``persuasive arguments theory'', bounded confidence. 
%     Neuro evidence supports minimal group paradigm as really affecting
%     neural activity \cite{Cikara2014}, rebutting concerns that studies using
%     that paradigm that found negative influence were flawed due to temporary
%     nature of minimal groups \cite{Krizan2007,Mas2013}.
%     \begin{itemize}
%       \item Compare to computational model in \citeA{Bishop1974}.
%       \item Bounded confidence seems to not have a mechanism for group
%         polarization among interacting agents---they will always settle to
%         the average of the group (proof somewhere?)
%     \end{itemize}
%   \item Using mean-field methods to calculate dynamics for any given $\alpha$
%     and $\gamma$ so this CPU-expensive matched filtering isn't used.
% \end{itemize}


% \vskip 40pt
% \underline {Guidelines:}
% \begin{itemize}
% \itemsep0pt
%   \item 2 pages maximum (including figure and references)
%   \item Single spaced
%   \item Font size 12pt
%   \item 1-inch margins
%   \item Do not include author names for blind review and remove any information that may identify the author(s)
% \end{itemize}

% \underline {Include in Order:}
% \begin{itemize}
% \itemsep0pt
%   \item Paper title
%   \item 5 keywords
%   \item Expanded abstract (= main text)  including figure
%   \item References
% \end{itemize}

\bibliographystyle{apacite}

\bibliography{/Users/mt/workspace/papers/library.bib}

\end{document}
